\documentclass[a4paper,12pt]{article}


\usepackage{amssymb,amsmath,amsfonts,eurosym,geometry,ulem,caption,color,setspace,sectsty,comment,caption,pdflscape,subfigure,array,hyperref, here}


\usepackage{bm}
% figure
\usepackage[dvipdfmx]{graphicx}
%\usepackage{bmpsize}

\usepackage{booktabs}
\usepackage{siunitx}
\newcolumntype{d}{S[
    input-open-uncertainty=,
    input-close-uncertainty=,
    parse-numbers = false,
    table-align-text-pre=false,
    table-align-text-post=false
 ]}


\usepackage[bottom]{footmisc}

\usepackage[round]{natbib}
\bibliographystyle{plainnat}

\begin{document}

\title{Oasis or Mirage? \\ Trade Network Propagation of Deforestation}
\author{Tomoki Nishiyama\footnote{Department of Economics, the University of Tokyo. \\ e-mail address: nishiyama-tomoki@g.ecc.u-tokyo.ac.jp \\ I thank Yasuyuki Sawada (an advisor of this thesis), Michihiro Kandori, and all the participants of their seminar for helpful discussion. I am also grateful to Shotaro Beppu, Yutao Chen, and Yuto Iwamoto for meaningful comments and supports.}}
\date{January 2024}
\maketitle

\begin{abstract}
    Log production is destroying forests around the world and contributing to global warming. To address deforestation, various countries have begun to adopt policies to restrict log exports. However, even if one country restricts log exports, the countries that have been importing logs may increase their log production to compensate for their previous imports. It is also possible that other countries will increase their log exports to the log consuming countries. In this case, one country's log export restrictions could induce deforestation in other countries through trade networks. In this paper, we focus on two policies, Russia's log export restrictions in 2007 and Myanmar's log export ban in 2014, to demonstrate how these policies affect the log industry in other countries through trade networks. We find that when one country restricts log exports, log production increases in their previously main export destination. This result suggests that through a network of trade, one country's pro-environmental policies can cause environmental damage in other countries.
\end{abstract}

\textit{
    A king is wandering in the desert. There used to be a forest in this desert. However, due to the mass production of logs, the former forest has been lost and has become a desert. The once cool forest is now a scorching desert with no water to be found no matter how far one walks. The king, who is about to collapse, finds something like water in the distance. The king says, "There is an oasis spreading out over there. There must be water. There must be trees. We can regenerate the forest from there." And he starts running. But, is that water really an oasis? Is it not a mirage?
}
\section{Introduction}
Deforestation is an urgent issue on this planet: 50\% of forests has already been lost since 1990 (DATA). Deforestation is causing devastation of the global environment. Deforestation accounts for 12-20\% of global greenhouse gas emissions, as it reduces the capacity to absorb carbon dioxide through plants' photosynthesis and contributes to climate change. Deforestation also causes loss of biodiversity because forests are home to a wide variety of plants and animals. The extraction of timber to produce forest products is one of the main causes of deforestation. To address this serious problem, some countries have banned log exports since the 1980s. For example, Myanmar, which has destroyed half of its tropical forests since 1960 and worked as the world's third largest supplier (DATA), banned log exports in 2014. \\

However, to measure the impact of environmental policies that target global issues, it is necessary to discuss how those policies affect other countries. Even a pro-environmental policy that protect forests in one's own country may cause new deforestation in other countries, thereby weakening the effectiveness of the policy on a global scale. To gain insight into this issue, this paper investigates how a country's forest protection policy can propagate deforestation to its trading partners through trade networks. I answer this critical question by examining two policies, Russia's restriction on log exports in 2007 and Myanmar's prohibition of log exports in 2014, respectively to check the impacts on the other countries' log production and exports.\\

Let us consider a simple case study. Myanmar prohibited their log exports in 2014. Until 2014, Myanmar was the third largest suppliers of timbers in the world. How would pro-environmental policy of major log supplies affect the other countries? First, trade partner countries, such as China or India, who imported lots of logs from Myanmar might extract more timbers in their own forests to compensate the supply shocks in the domestic log market. As a result, forests in Myanmar's export destination might be reduced due to Myanmar's pro-environmental policy. Second, the other countries, such as Congo or Indonesia, who shared the same export destination as Myanmar, might increase their exports to such destinations to compensate the reduction in Myanmar's exports. Then, forests in these supplier countries might be exploited due to Myanmar's export ban policy. As these stories, one country's policy to protect their own forest induces the other countries' deforestation through trade flow networks. Then, such network spillovers weaken the worldwide effect of the pro-environmental policy.\\

To empirically discuss this network propagation, I examine two historical policies, the 2007 Russian log export restriction and the 2014 Myanmar log export ban, respectively, and measure their effects on log production and exports in other countries. These two policy interventions have several useful features that make them ideal natural experiments for studying spillover effects. First, Russia and Myanmar were major suppliers of logs until their exports were restricted. Russia developed the largest coniferous forest in the world, the Siberian "taiga", which accounted for half of the world's supply of softwood lumber. Myanmar logged half of its tropical rainforests and exported teak logs to the world, accounting for 10\% of the world's tropical log supply. Therefore, a policy of export restrictions in these two major supplying countries would cause major supply shocks to other countries. Second, for countries other than these two, these policies were unpredictable exogenous shocks. Thus, the characteristics of these policies allow us to examine how environmental protection policies in one country affect forestry in other countries.\\

Using annual trade flow and production data of forestry industries, I employ shift-share design to empirically demonstrate effects of supply shock triggered by the ban of log exports. I regress outcome of each country on how much the country is exposed to the supply shock. The exposure to the shock is measured with interaction of two parts. The first part called ``shift'' is a measure of how much export-restricting country reduces their export to the rest of the world with the restriction. The second part called ``share'' is a measure of how much each country has been dependent on a specific market or industry before the restriction. Combining these two parts, we obtain measure of how much each country is exposed to the Russia or Myanmar's export restriction policy. By using this shift-share approach, I find that when log export restrictions are imposed in a country, log production increases in the country with the larger exposure to that supply shock, i.e., the country that was the export-restricting country's major export destination. This indicates that a country's log export restrictions contribute to deforestation in its major trading partners. On the other hand, for countries that exported to markets similar to that export-restricting country, there was no significant response in terms of production, export volume, or export value to the export-restriction shock. These results suggest that an pro-environmental policy in one country to protect its own forest has effect to propagate deforestation to their previous trade partners. This implicates that an efficacy of one country's forest protection can be crowded out by the spillover effects. It emphasizes that one government policy to protect its own forest is not sufficient to tackle with global environmental issue.\\

Although a large number of previous studies have discussed economics of deforestation, most studies have focused on relationship between policy interventions and deforestation within one country. For example, \cite{burgess2012political} describes how increase in political jurisdictions cause deforestation in Indonesia. \cite{abman2020does} discusses not only mechanism in which access to the timber markets increases log production but also mechanism in which trade openness to the agricultural market affects people's decision on land usage. \cite{harstad2023contingent} explains how trade agreement affects firms' decision on how much they exploit the forests. Nevertheless, the previous literature has not figured out how one country's policy to reduce deforestation propagates deforestation to the other countries through international trade. \\

On the other hand, literature of environment economics has studies how trade propagates environmental pollution is ``exported'' from developed countries to developing countries as ``pollution heaven hypothesis''. Since developed countries are more strict on firms' pollution of their neighboring environment than developing countries, firms often relocate factories which emit pollution materials from developed countries to developing countries. For example, \cite{hanna2010us} demonstrates that US regulation on firms' air pollution caused regulated multinational firms to increase their foreign output and foreign assets. \cite{tanaka2022north} examines effect of US air quality standard on relocation of battery recycling factories to Mexico and on infant health in Mexico. As these paper discusses, pro-environmental regulation in one country may propagate environmental issue to the other countries. \\

This paper contributes to fill the gap between literature on deforestation and that on pollution heaven hypothesis by empirically shows trade network propagates deforestation with one country's regulation. This study will provide critical evidence in discussing how to solve the issue of deforestation. That is because it is necessary to globally maintain forests and increase how much carbon dioxide can be absorbed in a global scale when we tackle international environment issue such as climate change. Protection of forests in one country at the expense of another country's forest does not contribute to reduction in global carbon dioxide. The results of this paper emphasizes this important implication. \\

As for the empirical method to examine propagation of a supply shock, I refer to recent growing literature of network analysis in economics. Acemoglu firstly introduces the analysis of how shock on an agent propagates to the other agents throughout network between agents. Based on this strand, various papers discuss how a shock to a specific firms affects the other firms that do not directly suffer from the shock through production supply chain. For example, \cite{barrot2016input} and \cite{carvalho2021supply} examine how shocks of the Eastern Japan Great Earthquake affects firms that are located far from the damaged area through their supply chain networks. I follow shift-share approach of \cite{huneeus2018production} to empirically show the network propagation effects of one country's pro-environmental policy on the other countries.\\

The remainder of this paper is organized as follows. Section 2 provides an overview of two trade restriction policies (Russia in 2007 and Myanmar in 2014) and explains how these policies are exogeneous shocks for other countries. Section 3 describes how I construct trade network flow data for forest products, and Section 4 presents a sketch of the network data to give an overall picture of the trade network. Based on some basic facts explained in Section 4, Section 5 presents a simple model for constructing hypotheses on how one country's policy affects other countries. Section 6 describes the identification strategy for empirically testing the theoretical hypothesis. Section 7 presents the empirical results and Section 8 discusses their interpretation. Finally, Section 9 presents the conclusions. Extensions of the theoretical model and robustness checks of the empirical results are in the appendix.\\

\cite{acemoglu2012network}

\section{Russia and Myanmar's Restriction Policy}

\section{Trade Flow Data}

\section{Sketch of the Trade Network of Logs}

\section{A Model of Deforestation Propagation}

\section{Identification Strategy}

\section{Estimation Results of Network Propagation}

\section{Discussion}

\section{Conoclusion}

\bibliography{deforestation_reference}

\end{document}