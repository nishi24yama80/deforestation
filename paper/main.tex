\documentclass[a4paper,12pt]{article}


\usepackage{amssymb,amsmath,amsfonts,eurosym,geometry,ulem,caption,setspace,sectsty,comment,caption,pdflscape,subfigure,array,hyperref, here}


\usepackage{bm}
% figure
\usepackage[dvipdfmx]{graphicx}
%\usepackage{bmpsize}

\usepackage{booktabs}
\usepackage{siunitx}
\newcolumntype{d}{S[
    input-open-uncertainty=,
    input-close-uncertainty=,
    parse-numbers = false,
    table-align-text-pre=false,
    table-align-text-post=false
 ]}


\usepackage[bottom]{footmisc}

\usepackage[round]{natbib}
\bibliographystyle{plainnat}

\begin{document}

\title{Oasis or Mirage? \\ Trade Network Propagation of Deforestation}
\author{Tomoki Nishiyama\footnote{Department of Economics, the University of Tokyo. \\ e-mail address: nishiyama-tomoki@g.ecc.u-tokyo.ac.jp \\ I thank Yasuyuki Sawada (a main advisory professor of this thesis), Michihiro Kandori, and all the participants of their seminar for helpful discussion. I am also grateful to Shotaro Beppu, Yutao Chen, and Yuto Iwamoto for meaningful comments and supports.}}
\date{January 2024}
\maketitle

\begin{abstract}
    Log production is destroying forests around the world and contributing to global warming. To address deforestation, various countries have begun to adopt policies to restrict log exports. However, even if one country restricts log exports, the countries that have been importing logs may increase their log production to compensate for their previous imports. It is also possible that other countries will increase their log exports to the log consuming countries. In this case, one country's log export restrictions could induce deforestation in other countries through trade networks. In this paper, Ifocus on two policies, Russia's log export restrictions in 2007 and Myanmar's log export ban in 2014, to demonstrate how these policies affect the log industry in other countries through trade networks. We find that when one country restricts log exports, log production increases in their previously main export destination. This result suggests that through a network of trade, one country's pro-environmental policies can cause environmental damage in other countries.
\end{abstract}

\textit{
    A king is wandering in the desert. There used to be a forest in this desert. However, due to the mass production of logs, the former forest has been lost and has become a desert. The once cool forest is now a scorching desert with no water to be found no matter how far one walks. The king, who is about to collapse, finds something like water in the distance. The king says, "There is an oasis spreading out over there. There must be water. There must be trees. We can regenerate the forest from there." And he starts running. But, is that water really an oasis? Is it not a mirage?
}
\section{Introduction}
Deforestation is an urgent issue on this planet: 50\% of forests has already been lost since 1990 (DATA). Deforestation is causing devastation of the global environment. Deforestation accounts for 12-20\% of global greenhouse gas emissions, as it reduces the capacity to absorb carbon dioxide through plants' photosynthesis and contributes to climate change. Deforestation also causes loss of biodiversity because forests are home to a wide variety of plants and animals. The extraction of timber to produce forest products is one of the main causes of deforestation. To address this serious problem, some countries have banned log exports since the 1980s. For example, Myanmar, which has destroyed half of its tropical forests since 1960 and worked as the world's third largest supplier (DATA), banned log exports in 2014. \\

However, to measure the impact of environmental policies that target global issues, it is necessary to discuss how those policies affect other countries. Even a pro-environmental policy that protect forests in one's own country may cause new deforestation in other countries, thereby weakening the effectiveness of the policy on a global scale. To gain insight into this issue, this paper investigates how a country's forest protection policy can propagate deforestation to its trading partners through trade networks. I answer this critical question by examining two policies, Russia's restriction on log exports in 2007 and Myanmar's prohibition of log exports in 2014, respectively to check the impacts on the other countries' log production and exports.\\

Let us consider a simple case study. Myanmar prohibited their log exports in 2014. Until 2014, Myanmar was the third largest suppliers of timbers in the world. How would pro-environmental policy of major log supplies affect the other countries? First, trade partner countries, such as China or India, who imported lots of logs from Myanmar might extract more timbers in their own forests to compensate the supply shocks in the domestic log market. As a result, forests in Myanmar's export destination might be reduced due to Myanmar's pro-environmental policy. Second, the other countries, such as Congo or Indonesia, who shared the same export destination as Myanmar, might increase their exports to such destinations to compensate the reduction in Myanmar's exports. Then, forests in these supplier countries might be exploited due to Myanmar's export ban policy. As these stories, one country's policy to protect their own forest induces the other countries' deforestation through trade flow networks. Then, such network spillovers weaken the worldwide effect of the pro-environmental policy.\\

To empirically discuss this network propagation, I examine two historical policies, the 2007 Russian log export restriction and the 2014 Myanmar log export ban, respectively, and measure their effects on log production and exports in other countries. These two policy interventions have several useful features that make them ideal natural experiments for studying spillover effects. First, Russia and Myanmar were major suppliers of logs until their exports were restricted. Russia developed the largest coniferous forest in the world, the Siberian ``taiga'', which accounted for half of the world's supply of softwood lumber. Myanmar logged half of its tropical rainforests and exported teak logs to the world, accounting for 10\% of the world's tropical log supply. Therefore, a policy of export restrictions in these two major supplying countries would cause major supply shocks to other countries. Second, for countries other than these two, these policies were unpredictable exogenous shocks. Thus, the characteristics of these policies allow us to examine how environmental protection policies in one country affect forestry in other countries.\\

Using annual trade flow and production data of forestry industries, I employ shift-share design to empirically demonstrate effects of supply shock triggered by the ban of log exports. I regress outcome of each country on how much the country is exposed to the supply shock. The exposure to the shock is measured with interaction of two parts. The first part called ``shift'' is a measure of how much export-restricting country reduces their export to the rest of the world with the restriction. The second part called ``share'' is a measure of how much each country has been dependent on a specific market or industry before the restriction. Combining these two parts, Iobtain measure of how much each country is exposed to the Russia or Myanmar's export restriction policy. By using this shift-share approach, I find that when log export restrictions are imposed in a country, log production increases in the country with the larger exposure to that supply shock, i.e., the country that was the export-restricting country's major export destination. This indicates that a country's log export restrictions contribute to deforestation in its major trading partners. On the other hand, for countries that exported to markets similar to that export-restricting country, there was no significant response in terms of production, export volume, or export value to the export-restriction shock. These results suggest that an pro-environmental policy in one country to protect its own forest has effect to propagate deforestation to their previous trade partners. This implicates that an efficacy of one country's forest protection can be crowded out by the spillover effects. It emphasizes that one government policy to protect its own forest is not sufficient to tackle with global environmental issue.\\

Although a large number of previous studies have discussed economics of deforestation, most studies have focused on relationship between policy interventions and deforestation within one country. For example, \cite{burgess2012political} describes how increase in political jurisdictions cause deforestation in Indonesia. \cite{abman2020does} discusses not only mechanism in which access to the timber markets increases log production but also mechanism in which trade openness to the agricultural market affects people's decision on land usage. \cite{harstad2023contingent} explains how trade agreement affects firms' decision on how much they exploit the forests. Nevertheless, the previous literature has not figured out how one country's policy to reduce deforestation propagates deforestation to the other countries through international trade. \\

On the other hand, literature of environment economics has studies how trade propagates environmental pollution is ``exported'' from developed countries to developing countries as ``pollution heaven hypothesis''. Since developed countries are more strict on firms' pollution of their neighboring environment than developing countries, firms often relocate factories which emit pollution materials from developed countries to developing countries. For example, \cite{hanna2010us} demonstrates that US regulation on firms' air pollution caused regulated multinational firms to increase their foreign output and foreign assets. \cite{tanaka2022north} examines effect of US air quality standard on relocation of battery recycling factories to Mexico and on infant health in Mexico. As these paper discusses, pro-environmental regulation in one country may propagate environmental issue to the other countries. \\

This paper contributes to fill the gap between literature on deforestation and that on pollution heaven hypothesis by empirically shows trade network propagates deforestation with one country's regulation. This study will provide critical evidence in discussing how to solve the issue of deforestation. That is because it is necessary to globally maintain forests and increase how much carbon dioxide can be absorbed in a global scale when Itackle international environment issue such as climate change. Protection of forests in one country at the expense of another country's forest does not contribute to reduction in global carbon dioxide. The results of this paper emphasizes this important implication. \\

As for the empirical method to examine propagation of a supply shock, I refer to recent growing literature of network analysis in economics. \cite{acemoglu2012network} firstly introduces the analysis of how shock on an agent propagates to the other agents throughout network between agents. Based on this strand, various papers discuss how a shock to a specific firms affects the other firms that do not directly suffer from the shock through production supply chain. For example, \cite{barrot2016input} and \cite{carvalho2021supply} examine how shocks of the Eastern Japan Great Earthquake affects firms that are located far from the damaged area through their supply chain networks. I follow shift-share approach of \cite{huneeus2018production} to empirically show the network propagation effects of one country's pro-environmental policy on the other countries.\\

The remainder of this paper is organized as follows. Section 2 describes how I construct trade network flow data for forest products. Section 3 provides an overview of two trade restriction policies (Russia in 2007 and Myanmar in 2014) and explains how these policies are exogeneous shocks for other countries. Section 4 presents a sketch of the network data to give an overall picture of the trade network. Based on some basic facts explained in Section 4, Section 5 presents a simple model for constructing hypotheses on how one country's policy affects other countries. Based on the theoretical predictions, Section 6 and 7 presents empirical results. Section 6 describes impacts of export restriction policy on the country's log trade partners. Then, Section 7 presents impacts on trade competitors that supply logs to the Russia's (or Myanmar's) trade partners. Section 8 discusses interpretation of the empirical results. Finally, Section 9 presents the conclusions. Extensions of the theoretical model and robustness checks of the empirical results are in the appendix.\\


\section{Trade Flow Data}
My empirical analysis uses the Food and Agriculture Organization of the United Nations database (hereafter FAOSTAT) as the main data set. FAOSTAT provides annual production and trade statistics for forest products. FAOSTAT is derived from two data sources. The first is annual trade flow data between countries. FAOSTAT provides annual data on the value and volume of imports and exports at the country level by product type, such as roundwood (coniferous roundwood, non-coniferous non-tropical roundwood, and non-coniferous tropical roundwood, respectively), saw wood, plywood, and pulp. The dataset covers 173 countries and regions from 1997 to 2018. For example, it provides information on the amount of non-coniferous tropical roundwood Myanmar exported to China in each year. The second is annual production and trade data at the country level. FAOSTAT provides annual country-level data on production, export value, export volume, import value, and import volume. This dataset covers annual data for 244 countries and regions from 1961 to 2022. For example, the dataset includes annual information on how much coniferous roundwood the Russian Federation developed in each year. These annual data on trade flows and production data are combined to build the main data set.\\

In addition to FAOSTAT forestry data, I employ two other data sources. First, I use intra-country bilateral distance data to estimate the impact of distance on trade flows. The distance data are borrowed from the GeoDist database provided by the Centre d'Etudes Prospectives et d'Informations Internationales (CEPII). I use bilateral distance between centers of gravity weighted by the population of each country. Second, Iuse annual GDP data from the World Bank database. I provide descriptive evidence of this dataset in the next section.\\

Before presenting empirical results, I define key terms used in empirical part of this study. The following parts frequently use the term ``roundwood'', which corresponds to ``industrial roundwood'' defined by FAOSTAT. The term ``roundwood'' refers to total volume or value of all roundwood, excluding wood fuels. Thus, ``roundwood'' contains sawlogs and veneer logs; pulpwood, round and split; and other industrial roundwood. In the dataset, roundwood are classified into three categories: coniferous roundwood, non-coniferous non-tropical roundwood, and non-coniferous tropical roundwood.


\section{Russia and Myanmar's Restriction Policy}
This paper examines how one country's environmental policies to protect forests, namely Russia's log export restriction in 2007 and Myanmar's log export ban in 2014, affected the other countries through trade networks. In this section, I briefly summarize these two policies and present qualitative evidence supporting my assumption that these two policies were exogeneous supply shocks to other countries.

\subsection{Russia's log export restriction in 2007}
Russia developed the largest coniferous forest in the world, the Siberian ``taiga'', which accounted for half of the world's supply of coniferous wood.  To cope with the deforestation and to promote domestic wood processing industry, Russian government imposed an export tax of 6.5\% in July 2007, rising to 20\% on July 2007 and 25\% in April 2008 on logs exported from Russia. This restriction on log exports was introduced due to the Russian domestic situation. Additionally, as METI points out, the new Russian Forest Code, which was base of these export tax policies, was founded in December 2006. This haste of policy interventions supports my assumption that the policy was exogeneous shock for the rest of the world. \\

Fig \ref{fig:shock_russia} shows significance of this Russian supply shock. Panel A represents time-series of import price, import quantity, and import value for coniferous roundwood. First, let us see Panel A. Each gray line represents the average price at which roundwood produced in each country is imported into the rest of the world. The blue line represents the average price at which roundwood produced in Russia is imported into countries other than Russia. In the same way, Panels B and C show the average import volume and total import value, respectively, of roundwood produced in each country and imported into the rest of the world.

Panel A shows that Russia's restriction on log exports in 2007 increases the average price of roundwood imports from Russia about 1.1 $\sim$ 1.4 times. This increase is consistent with the export tax rate. In addition to the effect on price, Panel B describes that the policy dramatically reduces the average quantity of roundwood imports from Russia by almost half. Panel C shows the policy leads to a rapid decrease in value of imports from Russia. In other words, the export restriction policy in 2007 has significant impact on the supply from Russia to the rest of the world market. 

\begin{figure}[H]
    \centering
    \caption{Impacts of Russia's Export Ban on Coniferous Roundwood Imports}
    \subfigure[][Import Price]{\includegraphics*[bb=0 0 650 650, scale=0.2]{"~/OneDrive - The University of Tokyo/deforestation/fig/shock/Russia/Industrial_roundwood__coniferous__export_import__imp_price.png"}}
    \subfigure[][Import Quantity]{\includegraphics*[bb=0 0 650 650, scale=0.2]{"~/OneDrive - The University of Tokyo/deforestation/fig/shock/Russia/Industrial_roundwood__coniferous__export_import__imp_quant.png"}}
    \subfigure[][Import Total Value]{\includegraphics*[bb=0 0 650 650, scale=0.2]{"~/OneDrive - The University of Tokyo/deforestation/fig/shock/Russia/Industrial_roundwood__coniferous__export_import__imp_val.png"}}
    \caption*{\small{Each gray line shows annual average price (a), quantity (b), and total value (c) of coniferous roundwood imports from each country. Blue line corresponds average price, quantity, and total value of imports from Russian Federation. Vertical line shows 2007, the year of Russia's restriction of log exports.}}
    \label{fig:shock_russia}
\end{figure}

\subsection{Myanmar's log export ban in 2014}
Myanmar had been a great supplier of teak (non-coniferous and tropical wood) in the world until 2014. However, half of the tropical rain forest had been lost by 2014 due to the log production and agriculture. To stop this serious deforestation, Myanmar government decided to ban exports of logs in 2014. WTO describes this situation as ``A ban on exports of round logs has been in place since 1 April 2014, to preserve Myanmar's dwindling forest cover, in particular, teak forests. The export prohibition also applies to boule-cut logs, baulk-square timber, and confiscated timber.'' \\

We can consider this export ban as an exogenous supply shock for the rest of the world. That is because the prohibition was due to the situation of Myanmar's country, which is not related to the other countries' situation. Fig \ref{fig:shock_Myanmar} shows significance of the supply shock. Panel A represents time-series of import price, import quantity, and import value for non-coniferous and non-tropical roundwood. First, let us see Panel A. Each gray line represents the average price at which roundwood produced in each country is imported into the rest of the world. The blue line represents the average price at which roundwood produced in Myanmar is imported into countries other than Myanmar. In the same way, Panels B and C show the average import volume and total import value, respectively, of roundwood produced in each country and imported into the rest of the world.\\

Panel A shows that Myanmar's export ban in 2014 did not have much impact on the average price of roundwood imports from Myanmar. Panel B, on the other hand, describes that the policy dramatically reduces the average quantity of roundwood imports from Myanmar from the third largest supplier. According to Panel C, the decrease in import volume leads to a decrease in value of imports from Myanmar. In other words, the export ban policy has significant impact on the supply from Myanmar to the rest of the world market. 

\begin{figure}[H] 
    \centering
    \caption{Impacts of Myanmar's Export Restriction on Non-coniferous Tropical Roundwood Imports}
    \subfigure[][Import Price]{\includegraphics*[bb=0 0 650 650, scale=0.2]{"~/OneDrive - The University of Tokyo/deforestation/fig/shock/Myanmar/Industrial_roundwood__non-coniferous_tropical__export_import__imp_price.png"}}
    \subfigure[][Import Quantity]{\includegraphics*[bb=0 0 650 650, scale=0.2]{"~/OneDrive - The University of Tokyo/deforestation/fig/shock/Myanmar/Industrial_roundwood__non-coniferous_tropical__export_import__imp_quant.png"}}
    \subfigure[][Import Total Value]{\includegraphics*[bb=0 0 650 650, scale=0.2]{"~/OneDrive - The University of Tokyo/deforestation/fig/shock/Myanmar/Industrial_roundwood__non-coniferous_tropical__export_import__imp_val.png"}}
    \caption*{\small{Each gray line shows annual average price (a), quantity (b), and total value (c) of non-coniferous tropical roundwood imports from each country. Blue line corresponds average price, quantity, and total value of imports from Myanmar. Vertical line shows 2014, the year of Myanmar's ban of log exports.}}
    \label{fig:shock_Myanmar}
\end{figure}

\section{Sketch of the Trade Network of Logs}
This section provides descriptive statistics to convey whole picture of the trade network data. First, I show cross-sectional summary statistics of roundwoods in 2018 with Table \ref{tab:summary_stat}. The summary statistics of roundwood export / import prices are presented in Table \ref{tab:summary_stat_price} in the Appendix. Based on these summary statistics, I presents several facts which describe the trade networks.\\

\begin{table}[htbp]
    \caption{Cross-sectional Summary Statistics on Roundwood in 2018}
    \centering
    \begin{tabular}{lcccccc}
        \tabularnewline \midrule \midrule
        variable & n & mean & sd & min & median & max\\
        \midrule
        \emph{Production Quantity ($m^3$)}\\
        Coniferous & 137 & 8858345 & 32249753 & 0 & 177000.0 & 294955296\\
        
        Non-coniferous & 176 & 5888862 & 20991251 & 0 & 468035.5 & 152244540\\
        
        \midrule
        \emph{Export Quantity ($m^3$)}\\
Coniferous & 164 & 935338.63 & 4136801.64 & 0 & 319.0 & 45779114\\

Non-coniferous Non-tropical & 166 & 306597.55 & 1377179.91 & 0 & 1593.5 & 14389012\\

Non-coniferous Tropical & 160 & 49025.86 & 192353.46 & 0 & 266.5 & 1728372\\
\midrule
\emph{Export Value ($1000$USD)}\\
Coniferous & 164 & 49572.77 & 219250.38 & 0 & 17.0 & 2426291\\

Non-coniferous Non-tropical & 166 & 26410.50 & 91635.86 & 0 & 485.5 & 848946\\

Non-coniferous Tropical & 160 & 17798.10 & 66185.29 & 0 & 114.0 & 531772\\
\midrule
\emph{Import Quantity ($m^3$)}\\
Coniferous & 172 & 1028691.20 & 7421256.74 & 0 & 5074.5 & 94999406\\

Non-coniferous Non-tropical & 173 & 194535.12 & 888875.56 & 0 & 822.0 & 9742233\\

Non-coniferous Tropical & 167 & 53211.02 & 603747.78 & 0 & 0.0 & 7794113\\
\midrule
\emph{Export Value ($1000$USD)}\\
Coniferous & 172 & 62749.98 & 452696.46 & 0 & 310.0 & 5794961\\

Non-coniferous Non-tropical & 173 & 26954.62 & 183700.13 & 0 & 155.0 & 2371915\\

Non-coniferous Tropical & 167 & 18924.81 & 217633.80 & 0 & 0.0 & 2810876\\

\midrule
\emph{Number of Export Links}\\
Coniferous & 164 & 6.835 & 11.375 & 0 & 1 & 59\\

Non-coniferous Non-tropical & 166 & 8.343 & 12.826 & 0 & 2 & 90\\

Non-coniferous Tropical & 160 & 3.481 & 5.428 & 0 & 1 & 27\\
\midrule
\emph{Number of Import Links}\\
Coniferous & 172 & 6.512 & 8.703 & 0 & 2 & 48\\

Non-coniferous Non-tropical & 173 & 7.919 & 12.083 & 0 & 2 & 89\\
Non-coniferous Tropical & 167 & 3.305 & 7.181 & 0 & 0 & 64\\
\midrule \midrule
        \end{tabular}
        \label{tab:summary_stat}
 \end{table}


\textit{\textbf{Fact 1:} Most of the roundwood production are supplied to the domestic market.} \\
Table \ref{tab:summary_stat} shows that around 10\% of the roundwood production are used for international trade and the rest 90\% are consumed in the domestic market. Figure \ref{tab:summary_stat_price} shows the long term trend of production and export of roundwoods. In any period, around 90 \% of the production is consumed in the domestic market\footnote{There is a jump in the export quantity of non-coniferous non-tropical roundwood in 2004. This might be due to errors of the original FAOSTAT dataset.}.

\begin{figure}[htbp]
    \centering
    \caption{Roundwood World Production and Export}
    \subfigure[][Production Quantity $(m^3)$]{\centering \includegraphics*[bb=0 0 1000 1000, scale=0.2]{"~/OneDrive - The University of Tokyo/deforestation/fig/summary_stat/production.png"}}
    \subfigure[][Export Quantity $(m^3)$]{\centering \includegraphics*[bb=0 0 1000 1000, scale=0.2]{"~/OneDrive - The University of Tokyo/deforestation/fig/summary_stat/exp_quant.png"}}
    \caption*{\small{The left panel is a time-series plot of total production and the right is that of total export.}}
    \label{fig:time_series_prod_exp}
\end{figure}

\textit{\textbf{Fact 2:} Roundwood is used as intermediate inputs of forestry products.} \\
Roundwood is used as intermediate input goods of various forestry product, such as sawn wood, plywood, pulp, and wood chip. Table \ref{tab:intermediate_input} shows relationship between quantity of forestry products made in each country and quantity of roundwood produced for domestic market and roundwood imported from foreign suppliers. As a control, I include lagged GDP on the right-hand side. There are positive relationship between quantity of roundwoods and quantity of forestry products. This result is consistent with the fact that roundwoods are used as intermediate inputs of forestry products.

\begin{table}[htbp]
    \caption{Production of Forestry Goods and Usage of Logs}
    \centering
    \begin{tabular}{lcccc}
       \tabularnewline \midrule \midrule
       Dependent Variables:     & Sawn wood                    & Plywood                     & Pulp                    & Wood Chip\\  
       Model:                   & (1)                     & (2)                         & (3)                     & (4)\\  
       \midrule
       \emph{Variables}\\
       Conif Domestic          & 0.2160$^{***}$          & 0.0567                      & 0.0378                  & -0.0339\\   
                                & (0.0651)                & (0.0514)                    & (0.0379)                & (0.2233)\\   
       Conif Imp               & 1.659$^{**}$            & 0.7917$^{**}$               & 0.3218$^{**}$           & 1.039$^{*}$\\   
                                & (0.6541)                & (0.3536)                    & (0.1473)                & (0.5378)\\   
       Non-conif Domestic       & -0.0033                 & -0.0903                     & 0.0080                  & 0.1929\\   
                                & (0.0879)                & (0.0607)                    & (0.0482)                & (0.1786)\\   
     Non-conif Non-trop Imp   & -0.2427                 & -0.2589                     & 0.2553$^{**}$           & 0.7318\\   
                                & (0.6352)                & (0.2551)                    & (0.1219)                & (0.9474)\\  
       Non-conif Trop Imp      & 0.5227                  & 1.208                       & -0.3307                 & 2.094$^{*}$\\   
                                & (2.061)                 & (1.079)                     & (0.3494)                & (1.216)\\    
       Lagged GDP                & $2.46\times 10^{-6}$    & $1.86\times 10^{-6}$$^{*}$  & $2.83\times 10^{-7}$    & $-1.06\times 10^{-7}$\\    
                                & ($1.95\times 10^{-6}$)  & ($1.06\times 10^{-6}$)      & ($5.22\times 10^{-7}$)  & ($1.79\times 10^{-6}$)\\    
       \midrule
       \emph{Fixed-effects}\\
       Country                     & Yes                     & Yes                         & Yes                     & Yes\\  
       Year          & Yes                     & Yes                         & Yes                     & Yes\\  
       \midrule
       \emph{Fit statistics}\\
       Observations             & 2,812                   & 2,812                       & 1,869                   & 1,759\\  
       R$^2$                    & 0.97680                 & 0.96827                     & 0.98507                 & 0.86730\\  
       Within R$^2$             & 0.73537                 & 0.72402                     & 0.33070                 & 0.21385\\  
       \midrule \midrule
    \end{tabular}
    \caption*{\small{Conif : coniferous, Trop: tropical, Imp : import quantity.Clustered (area) standard-errors are reported in parentheses. Signif. Codes: ***: 0.01, **: 0.05, *: 0.1}}
    \label{tab:intermediate_input}
 \end{table}

 \textit{\textbf{Fact3:} The distribution of active export / import links (links with positive export / import quantity) is right-skewed.}\\
 I construct bipartite graph consisted of around 160 importer countries and 160 exporter countries. However, not all the links between potential importer countries and exporter countries are not active. In other words, some countries do not export roundwood at all to some other countries. To examine the number of active links of roundwood trade, I present histograms in Figure \ref{fig:num_of_trade_links}. First, in both export and import of any roundwood, the distribution has a long tail on the right side. In addition, the number of countries with zero links are larger in exports than in imports. This is consistent with the uneven distribution of forest resources over countries due to different climate or areas.

 \begin{figure}[H]
    \centering
    \caption{Number of Roundwood Trade Links in 2018}
    \subfigure[][Export Links]{\centering \includegraphics*[bb=0 0 1000 1000, scale=0.2]{"~/OneDrive - The University of Tokyo/deforestation/fig/num_links/num_links_exp.png"}}
    \subfigure[][Import Links]{\centering \includegraphics*[bb=0 0 1000 1000, scale=0.2]{"~/OneDrive - The University of Tokyo/deforestation/fig/num_links/num_links_imp.png"}}
    \caption*{\small{The left panel is a histogram of number of links for roundwood exports, and the right panel is a histogram of number of links for roundwood imports. The unit is country-level, and there are 164 countries for coniferous roundwood, 166 countries for non-coniferous non-tropical roundwood, and 160 countries for non-coniferous tropical roundwood.}}
    \label{fig:num_of_trade_links}
\end{figure}

\textit{\textbf{Fact4:} Intensive margin (average value of export / import per one active link) accounts for more than half of the variation of countries' export and import}.\\
Table \ref{tab:summary_stat} suggests that there is a large variation of export value and import value among countries. Since the value is a product of unit average price and quantity, we can decompose the variation of export and import value into variation of unit average price and variation of quantity. By following decomposition method proposed by \cite{bernard2009margins}, we decompose the variation by the following process:

\begin{align}
    \intertext{Since the total value of export (import) $y_i$ is a product of number of active links of export (import) $q_i$ and average export (import) value per link $\Bar{p}_i = y_i / q_i$, namely,}
    y_i = q_i \times \Bar{p}_i
    \intertext{By taking log, we get additive equation:}
    \log y_i = \log q_i + \log \Bar{p}_i
    \intertext{Thus, we can decompose the variation of $y_i$ to variation of $q_i$ and $\Bar{p}_i$ with the following cross-sectional regressions (\cite{bernard2009margins}),}
    \log q_i = \beta_1 \log y_i + \varepsilon_i \\
    \log \Bar{p}_i = \beta_2 \log y_i + \varphi_i \\
\end{align}

The results of this decomposition in 2018 are displayed in Table \ref{tab:decomp_exp} and \ref{tab:decomp_imp}. ``Links'' indicates a esimate of $\beta_1$ and ``Average'' indicates a estimate of $\beta_2$. These results suggest that around $50 \sim 60$ \% of the variation of the export (import) value is explained by the variation of average export (import) value per link. In other words, more than half of the variation of the export (import) comes from variation of the intensive margin.

\begin{table}[htbp]
    \centering
    \caption{Cross-sectional Decomposition of Countries' Log Export Value}
    \begin{tabular}[t]{lcccccc}
    \toprule
    \textit{log type:} &  \multicolumn{2}{c}{Coniferous}&  \multicolumn{2}{c}{Non-conif Non-tropical}&  \multicolumn{2}{c}{Non-conif tropical}\\
    \midrule
      & (1) & (2) & (3) & (4) & (5) & (6)\\
      \textit{Dependent Var:}& Links & Average & Links & Average & Links & Average \\
    \midrule
    log(Export Value)  & \num{0.390}*** & \num{0.610}*** & \num{0.412}*** & \num{0.588}*** & \num{0.379}*** & \num{0.621}***\\
    & (\num{0.010}) & (\num{0.010}) & (\num{0.010}) & (\num{0.010}) & (\num{0.013}) & (\num{0.013})\\
   \midrule
   Num.Obs. & \num{95} & \num{95} & \num{120} & \num{120} & \num{102} & \num{102}\\
   R2 & \num{0.899} & \num{0.956} & \num{0.919} & \num{0.959} & \num{0.894} & \num{0.957}\\
   \bottomrule
   \multicolumn{7}{l}{\rule{0pt}{1em}* p $<$ 0.1, ** p $<$ 0.05, *** p $<$ 0.01}\\
   \end{tabular}
   \label{tab:decomp_exp}
   \end{table}

\begin{table}[htbp]
    \centering
    \caption{Cross-sectional Decomposition of Countries' Log Import Value}
    \begin{tabular}[t]{lcccccc}
    \toprule
    \textit{log type:} &  \multicolumn{2}{c}{Coniferous}&  \multicolumn{2}{c}{Non-conif Non-tropical}&  \multicolumn{2}{c}{Non-conif tropical}\\
    \midrule
      & (1) & (2) & (3) & (4) & (5) & (6)\\
      \textit{Dependent Var:}& Links & Average & Links & Average & Links & Average \\
    \midrule
    log(Import Value) & \num{0.389}*** & \num{0.611}*** & \num{0.437}*** & \num{0.563}*** & \num{0.479}*** & \num{0.521}***\\
    & (\num{0.008}) & (\num{0.008}) & (\num{0.009}) & (\num{0.009}) & (\num{0.017}) & (\num{0.017})\\
   \midrule
   Num.Obs. & \num{141} & \num{141} & \num{136} & \num{136} & \num{80} & \num{80}\\
   R2 & \num{0.916} & \num{0.964} & \num{0.923} & \num{0.952} & \num{0.882} & \num{0.899}\\
   \bottomrule
   \multicolumn{7}{l}{\rule{0pt}{1em}* p $<$ 0.1, ** p $<$ 0.05, *** p $<$ 0.01}\\
   \end{tabular}
   \label{tab:decomp_imp}
   \end{table}

   \textit{\textbf{Fact5:} Trade flow is sensitive to the distance between importer and exporter.}.\\
   In general, as the distance between two countries increases, the volume of trade between them decreases (\cite{allen201813}). To examine this relationship in the trade of roundwood, I regress the following regression, what is called, ``Gravity Equation'', with panel data over 2001 to 2018:
   \begin{align}
    \log v_{i \to j, t} = \beta_0 + \beta_1 \log d_{i \to j} + \delta_{i, t} + \gamma_{j, t} + \varepsilon_{i,j,t}
   \end{align}
   where $v_{i \to j, t}$ represents export value from country $i$ to $j$ at year $t$ and $d_{i \to j}$ is a time in-variant bilateral distance between $i$ and $j$. Table \ref{tab:summary_stat} suggests that the elasticity of distance is $-0.7 \sim -2$, which indicates that 1\% increase in the distance between two countries results in a $-0.7 \sim -2$ \% decrease in the trade value between them. This value is consistent with \cite{allen201813}.
   
   \begin{table}[htbp]
    \caption{Cross-sectional Gravity Equation of Log Export Value}
    \centering
    \begin{tabular}{lccc}
       \tabularnewline \midrule \midrule
       Dependent Variable: & \multicolumn{3}{c}{log(Export Value)}\\
       \midrule
       \textit{log type: } & Coniferous & Non-conif Non-tropical & Non-conif Tropical\\
       Model:                             & (1)            & (2)            & (3)\\  
       \midrule
       \emph{Variables}\\
       log(distance)                         & -1.916$^{***}$ & -1.226$^{***}$ & -0.6735$^{***}$\\   
       & (0.1864)       & (0.1215)       & (0.0907)\\   
        \midrule
        \emph{Fixed-effects}\\
        year-exporter  & Yes            & Yes            & Yes\\  
        year-importer   & Yes            & Yes            & Yes\\  
        \midrule
        \emph{Fit statistics}\\
        Observations                       & 17,124         & 22,695         & 9,494\\  
        R$^2$                              & 0.51745        & 0.47173        & 0.60015\\  
        Within R$^2$                       & 0.21867        & 0.15971        & 0.06400\\  
        \midrule \midrule
        \multicolumn{4}{l}{\emph{Clustered (exporter country and importer country) standard-errors in parentheses}}\\
        \multicolumn{4}{l}{\emph{Signif. Codes: ***: 0.01, **: 0.05, *: 0.1}}\\
    \end{tabular}
 \end{table}

\section{A Model of Deforestation Propagation}
Based on the properties of the roundwood trade network, this section constructs a simple model of the roundwood trade, and then presents a hypothesis induced by prediction of the model. The basic framework is quantity competition, what is called, Cournot competition. In usual Cournot setting, firms decide quantity of supply to a market to maximize their profit. On the other hand, I extend this model to quantity competition among firms with assynmetric trade costs. This section describes the Cournot competition model as a game where $n$ countries supply homogeneous roundwood to one market. I introduce a general model at first, and then simplify the model to discuss the implication. \\

I would like to assume that there are n countries which has one market denoted by $M_i$, respectively. Thus, in total, there are $n$ markets $\{ M_1, ..., M_n \}$. In addition to the market, each country has one supplier firm $F_i$, and there are $n$ firms $\{ F_1, ..., F_n \}$ in total. Thus, this is a Cournot model where markets and firms belong to a bipartite graph. Every firm produces an homogeneous good to supply $n$ markets. Each firm choose a vector comsisted of quantity to supply each market, $\mathbf{q}_i = (q_{i1}, ..., q_{in})$ where $q_{ij}$ means quantity that firm $i$ supplies to market $j$. Every market has an identical inverse linear demand function $p_i = D(q_{1i}, ..., q_{ni}) = a_i - \sum_{j=1}^n q_{ji}$. Profit function of each firm is given by $\pi_i (\mathbf{q}_1, ..., \mathbf{q}_n) = \sum_{j = 1}^{n} ((a_j - \sum_{k=1}^n q_{kj}) q_{ij} - c_{i} (\sum_{j = 1}^{n} q_{ij} ) ^ 2 )$. Firms choose their strategy $\mathbf{q}_i$, that is, how much the firm produce and supply to each market, to maximize their profit. As \cite{bimpikis2019cournot} discusses, there is unique Nash equilibrium when this game has symmetric structure, that is, $a_1 = a_2 = ... = a_n = a$ and $c_1 = c_2 = ... = c_n = c$. On the other hand, the uniqueness of the equilibrium and the analytical solution is unknown for assymmetric cases. However, my interest is how assymmetric case where each firm has heterogeneous trade cost to the markets. To be more precise, I would like to discuss how assymmetric trade costs affect firms' decision when one of the firm exit from the markets. Therefore, Ineed to simplify this model from the general model. \\

From now on, Iwill consider a case where there are three countries. We assume that one of the three countries (country $1$) has both a market and a supplier, and the other two countries (country $2$ and country $3$) have only a supplier respectively. We can consider the country with both a market and a supplier as a country which produce roundwood and supply to Chinese market itself. On the other hand, the rest countries are developing countries that have plenty of forests to supply to the large country but that do not have a sufficiently large domestic market, such as Myanmar. Each of the three suppliers (Supplier $i$) chooses quantity $q_i$ to supply the market. The market of country $1$ has a inverse linear demand function
\begin{align}
    p = a - \sum_{i=1}^3 q_i
\end{align}

The suppliers are located in different countries, so they have different shipping cost to the market. We assume that the cost function is a quadratic shape $c_{i} q_i ^ 2$, where $c_i$ is the sum of production cost to produce the goods and trade cost to supply the goods to the market. If the production cost is the same among three firms, then the $c_1$ is smaller than $c_2$ and $c_3$ because the supplier of country $1$ is located in the same country as the market while suppliers in country $2$ and $3$ need to transport the product with bulk log carrier. The supplier $i$'s profit function is given by 
\begin{align}
    \pi_i (q_1, q_2, q_3) = \left( a - \sum_{j=1}^3 q_{j} \right) q_{i} - c_{i} q_i ^ 2 
\end{align}

Supplier $i$ choose the optimal supply quantity $q_i$ which maximizes $i$'s the profit. Thus, by first order conditions, the Nash equilibrium $q_1^*, q_2^*, q_3^*$ satisfies the following three equations:

\begin{align}
    2(c_1 + 1) q_1^* + q_2 ^* + q_2 ^ * + q_3 ^ * - a &= 0 \\
    2(c_2 + 1) q_2^* + q_2 ^* + q_3 ^ * + q_1 ^ * - a &= 0 \\
    2(c_3 + 1) q_3^* + q_2 ^* + q_1 ^ * + q_2 ^ * - a &= 0
\end{align}

Thus, Iobtain 
\begin{align}
    q_1^* = \frac{4 a c_{2} c_{3} + 2 a c_{2} + 2 a c_{3} + a}{8 c_{1} c_{2} c_{3} + 8 c_{1} c_{2} + 8 c_{1} c_{3} + 6 c_{1} + 8 c_{2} c_{3} + 6 c_{2} + 6 c_{3} + 4} \\
    q_2^* = \frac{4 a c_{1} c_{3} + 2 a c_{1} + 2 a c_{3} + a}{8 c_{1} c_{2} c_{3} + 8 c_{1} c_{2} + 8 c_{1} c_{3} + 6 c_{1} + 8 c_{2} c_{3} + 6 c_{2} + 6 c_{3} + 4} \\
    q_3^* = \frac{4 a c_{1} c_{2} + 2 a c_{1} + 2 a c_{2} + a}{8 c_{1} c_{2} c_{3} + 8 c_{1} c_{2} + 8 c_{1} c_{3} + 6 c_{1} + 8 c_{2} c_{3} + 6 c_{2} + 6 c_{3} + 4}
\end{align}

Next, let us consider the situation where country $3$ bans its log export and supplier $3$ exits from the market. Then, how will supplier $1$ and $2$ react? Since $q_3$ is fixed at $0$, the profit function of supplier $1$ and $2$ are 

\begin{align}
    \pi_i (q_1, q_2, 0) = (a - q_1 - q_2) q_{i} - c_{i} q_i ^ 2 
\end{align}

Thus, the Nash equilibrium after country $3$'s export ban $q_1^{**}, q_2^{**}$ satisfies the following two equations:

\begin{align}
    2(c_1 + 1) q_1^{**} + q_2 ^{**} - a &= 0 \\
    2(c_2 + 1) q_2^{**} + q_1 ^ {**} - a &= 0 
\end{align}

Thus, the new Nash equilibrium after the export ban policy is
\begin{align}
    q_1^{**} = \frac{2 a c_{2} + a}{4 c_{1} c_{2} + 4 c_{1} + 4 c_{2} + 3} \\
    q_2^{**} = \frac{2 a c_{1} + a}{4 c_{1} c_{2} + 4 c_{1} + 4 c_{2} + 3}
\end{align}

Finally, I would like to check how assymmetric trade cost affects each firm's equilibrium strategy (while assuming that production cost is identical among suppliers). As I can infer obviously from the equilibrium strategies, $c_i > c_j > c_k$ implies $q_i^* < q_j^* < q_k^*$ before one firm exits from the market and $c_i > c_j$ implies $q_i^{**} < q_j^{**}$ after one firm exits from the market. It implies that suppliers who need to pay large shipping cost to supply the market produce relatively small quantity. Our main interest is how different trade cost affects suppliers' reaction to the other supplier's exit. The difference in equilibrium supply quantity between before and after Country $3$'s export ban is given by:

\begin{align}
    \Delta_1 &= q_1^{**} - q_1^{*} \\
    \Delta_2 &= q_2^{**} - q_2^{*}
\end{align}

Intuitively speaking, Iexpect that $\Delta_i > \Delta_j$ if $c_i < c_j$. Since the form of equilibrium strategies is too complicated to analytical proove this intuitive relationship, I presents results of analytical simulation which support this intuitive property. Figure \ref{fig:cournot_simulation} displays simulation results where $a = 10$ (the intercept of the inverse demand function is $10$) and $c_3 = 1$ (supplier $3$'s cost to produce and supply $q_3$ units is $1$). I calculate $\Delta_1$ and $\Delta_2$ for a total of 121 combinations of $c_1 = 0, 1, ..., 10$ and $c_2 = 0, 1, ..., 10$. Each dot in the figure shows the magnitude relationship between $\Delta_1$ and $\Delta_2$ with its shape and the difference with its color. This figure implies that $\Delta_1 > \Delta_2$ if $c_2 > c_1$, that $\Delta_1 = \Delta_2$ if $c_2 = c_1$, and that that $\Delta_1 < \Delta_2$ if $c_2 < c_1$. In addition to this simulation, I confirm that this result is not sensitive to the value of $a$ and $c_3$. These results support the property of ``$\Delta_i > \Delta_j$ if $c_i < c_j$ ''. \\

\begin{figure}[H] 
    \centering
    \caption{Impacts of Russia's Export Restriction on Its Trade Partners' Roundwood Production}
    \includegraphics*[bb=0 0 900 700, scale=0.3]{"~/OneDrive - The University of Tokyo/deforestation/fig/cournot/trade_cost_exit.png"}
    \caption*{\small{The figure shows simulation results of $\Delta_1$ and $\Delta_2$ with $a = 10$ and $c_3 = 0$ for each $c_1 = 0, 1, ..., 10$ and $c_2 = 0, 1, ..., 10$. The square corresponds to $\Delta_1 < \Delta_2$, triangle corresponds to $\Delta_1 = \Delta_2$, and square corresponds to $\Delta_1 > \Delta_2$. The color of dots shows value of $\Delta_2 - \Delta_1$.}}
    \label{fig:cournot_simulation}
\end{figure}

Moreover, this property suggests that if supply shock happens to one market, then the supplier with the smallest transportation costs will increase its production and its supply to the market to the greatest extent. In other words, the suppliers located in the same country as the market will increase their production and supply by larger extent than the foreign suppliers. For example, if China, who imported a lot of roundwood from Russia before Russia's restriction on log export, experience reduction in imports from Russia by the restriction, then China itself increase production of roundwood and supply to the home market by larger extent than the suppliers in the other countries. In the following empirical sections, Iwill examine how the supply shock caused by Russia and Myanmar affects the other countries' production and trade flows. The model predicts that the impacts of the supply shock is larger for Russia's (or Myanmar's) direct trade partner, whose market experience reduction in supply, than the other countries. In other words, the reaction of countries who imported roundwood from Russia (Myanmar) increases their production rapidly while the other countries increase production towards such markets by less extent.



\section{Impacts on Export Destinations}
In this section, I examine how Russia's log export restrictions and Myanmar's log export ban affected its trade partners, respectively. For instance, Myanmar exported large number of teak logs (non-coniferous and tropical roundwood) to China, but how did China react to the Myanmar's supply shock in 2014? The quantity competition model predicts that China would increase production of non-coniferous roundwood to supply its home market due to the small trade cost. To empirically test this hypothesis, I employ a framework of potential outcome and apply generalized synthetic control method proposed by \cite{xu2017generalized}. I first describe the identification strategy and then present the estimation results.

\subsection{Identification strategy}

We would like to know how the Russia's (Myanmar's) export restriction (ban) policy would affect its trade partners, i.e., export destinations. In the extended Cournot model, such trade partners are expected to increase their production to supply the home market and compensate the reduction in its import from Russia (Myanmar) because such country's trade cost to ship to its home market is much lower than the other country's cost to ship to the market. \\

In order to apply the causal inference framework, I need to define a treatment in this problem. Here, Idefine a treatment as a decrease in imports from Russia (Myanmar) due to Russia's (Myanmar's) export restriction policy. If one country imported logs from Russia (Myanmar) before the supply shock, the country would be likely to continue the import from Russia (Myanmar) if Russia (Myanmar) did not implement export restriction policy. On the other hand, if one country did not import any logs from Russia (Myanmar) even before the supply shock, the country would not be likely to import from Russia (Myanmar) even without any restriction policy. Thus, as a proxy of the treatment, Iemploy a binary variable which indicates the country imported logs from Russia (Myanmar) before the restriction policy. We denote this binary variable with $T_i$, where $T_i$ is 1 if country $i$ imported coniferous roundwood (non-coniferous and non-tropical roundwood) from Russia (Myanmar) one year before the policy intervention, and is 0 otherwise. $T_i = 1$ implies that the country I would be directly exposed to the policy intervention of Russia (Myanmar) in terms of the reduction in imports from Russia (Myanmar). On the other hand, $T_i = 0$ implies that the country would not be directly exposed to the policy. The outcome of interest is country $i$'s production, import, export of roundwoods.\\

Since Ihave long-term trade panel data before and after the policy intervention, Ican implement difference-in-differences approach to estimate the average treatment effects on the treated units (ATT). However, one concern is the heterogeneity in trends between treatment group, that is, those who imported roundwood from Russia (Myanmar), and trend of control group, that is, those who did not import roundwood from Russia (Myanmar). For example, countries who imported logs from Russia (Myanmar) have different locational characteristics from control countries. To address this problem, I use synthetic difference-in-differences method proposed by \cite{xu2017generalized}. \cite{xu2017generalized} integrates synthetic control methods (\cite{abadie2010synthetic}) and linear fixed effects models of DID framework. This method infers counterfactuals for each treated unit with data of control group units by applying synthetic control method. In addition to that, \cite{xu2017generalized} extends the synthetic control approach to a case of multiple treated units.\\ 

In the estimation, I use country-year-level panel data from 1990 to 2018. We employ two-way fixed effects as additive components of the right-hand side. To obtain information on the uncertainty of the estimates, I calculate standard errors using parametric method with 1000 bootstrap sampling.

\subsection{Estimation Results}
\subsubsection*{Impacts of Russia's Log Export Restriction in 2007}

\begin{figure}[H] 
    \centering
    \caption{Impacts of Russia's Export Restriction on Its Trade Partners' Roundwood Production}
    \subfigure[][Coniferous]{\includegraphics*[bb=0 0 900 700, scale=0.2]{"~/OneDrive - The University of Tokyo/deforestation/fig/syntheticDID/Russia/ATT/Production_Industrial_roundwood__coniferous.png"}}
    \subfigure[][Non-coniferous]{\includegraphics*[bb=0 0 900 700, scale=0.2]{"~/OneDrive - The University of Tokyo/deforestation/fig/syntheticDID/Russia/ATT/Production_Industrial_roundwood__non-coniferous.png"}}
    \caption*{\small{The figure shows impacts of Russia's export restriction policy in 2007 on their trade partners' production volumne of coniferous roundwood (a) and non-coniferous roundwood. X axis represents ellapesed time from the policy intervention. Thus, t = 0 means 2007. Y axis is the estimates of ATT. The black lines represent the point estimates and gray shadows represnt 95\% confidence interval.}}
    \label{fig:att_prod_Russia}
\end{figure}

\begin{figure}[H] 
    \centering
    \caption{Impacts of Russia's Export Restriction on Its Trade Partners' Roundwood Import Volume}
    \subfigure[][Coniferous]{\includegraphics*[bb=0 0 650 650, scale=0.2]{"~/OneDrive - The University of Tokyo/deforestation/fig/syntheticDID/Russia/ATT/imp_quant_Coniferous.png"}}
    \subfigure[][Non-conif Non-tropical]{\includegraphics*[bb=0 0 650 650, scale=0.2]{"~/OneDrive - The University of Tokyo/deforestation/fig/syntheticDID/Russia/ATT/imp_quant_Non-coniferous_non-tropical.png"}}
    \subfigure[][Non-conif Tropical]{\includegraphics*[bb=0 0 650 650, scale=0.2]{"~/OneDrive - The University of Tokyo/deforestation/fig/syntheticDID/Russia/ATT/imp_quant_Non-coniferous_tropical.png"}}
    \caption*{\small{The figure shows impacts of Russia's export restriction policy in 2007 on their trade partners' import volumne of coniferous roundwood (a), non-coniferous non-tropical roundwood (b), and non-coniferous tropical roundwood (c). X axis represents ellapesed time from the policy intervention. Thus, t = 0 means 2007. Y axis is the estimates of ATT. The black lines represent the point estimates and gray shadows represnt 95\% confidence interval.}}
    \label{fig:att_imp_Russia}
\end{figure}


\begin{figure}[H] 
    \centering
    \caption{Impacts of Russia's Export Restriction on Its Trade Partners' Roundwood Production}
    \subfigure[][Coniferous]{\includegraphics*[bb=0 0 900 700, scale=0.2]{"~/OneDrive - The University of Tokyo/deforestation/fig/syntheticDID/Russia/CF/Production_Industrial_roundwood__coniferous.png"}}
    \subfigure[][Non-coniferous]{\includegraphics*[bb=0 0 900 700, scale=0.2]{"~/OneDrive - The University of Tokyo/deforestation/fig/syntheticDID/Russia/CF/Production_Industrial_roundwood__non-coniferous.png"}}
    \caption*{\small{The figure shows impacts of Russia's export restriction policy in 2007 on their trade partners' production volumne of coniferous roundwood (a) and non-coniferous roundwood. X axis represents ellapesed time from the policy intervention. Thus, t = 0 means 2007. Y axis is the estimates of ATT. The black lines represent the point estimates and gray shadows represnt 95\% confidence interval.}}
    \label{fig:cf_prod_Russia}
\end{figure}

\begin{figure}[H] 
    \centering
    \caption{Impacts of Russia's Export Restriction on Its Trade Partners' Roundwood Import Volume}
    \subfigure[][Coniferous]{\includegraphics*[bb=0 0 650 650, scale=0.2]{"~/OneDrive - The University of Tokyo/deforestation/fig/syntheticDID/Russia/CF/imp_quant_Coniferous.png"}}
    \subfigure[][Non-conif Non-tropical]{\includegraphics*[bb=0 0 650 650, scale=0.2]{"~/OneDrive - The University of Tokyo/deforestation/fig/syntheticDID/Russia/CF/imp_quant_Non-coniferous_non-tropical.png"}}
    \subfigure[][Non-conif Tropical]{\includegraphics*[bb=0 0 650 650, scale=0.2]{"~/OneDrive - The University of Tokyo/deforestation/fig/syntheticDID/Russia/CF/imp_quant_Non-coniferous_tropical.png"}}
    \caption*{\small{The figure shows impacts of Russia's export restriction policy in 2007 on their trade partners' import volumne of coniferous roundwood (a), non-coniferous non-tropical roundwood (b), and non-coniferous tropical roundwood (c). X axis represents ellapesed time from the policy intervention. Thus, t = 0 means 2007. Y axis is the estimates of ATT. The black lines represent the point estimates and gray shadows represnt 95\% confidence interval.}}
    \label{fig:cf_imp_Russia}
\end{figure}

\subsubsection*{Impacts of Myanmar's Log Export Restriction in 2007}

\begin{figure}[H] 
    \centering
    \caption{Impacts of Myanmar's Export Restriction on Its Trade Partners' Roundwood Production}
    \subfigure[][Coniferous]{\includegraphics*[bb=0 0 900 700, scale=0.2]{"~/OneDrive - The University of Tokyo/deforestation/fig/syntheticDID/Myanmar/ATT/Production_Industrial_roundwood__coniferous.png"}}
    \subfigure[][Non-coniferous]{\includegraphics*[bb=0 0 900 700, scale=0.2]{"~/OneDrive - The University of Tokyo/deforestation/fig/syntheticDID/Myanmar/ATT/Production_Industrial_roundwood__non-coniferous.png"}}
    \caption*{\small{The figure shows impacts of Myanmar's export restriction policy in 2014 on their trade partners' production volumne of coniferous roundwood (a) and non-coniferous roundwood. X axis represents ellapesed time from the policy intervention. Thus, t = 0 means 2014. Y axis is the estimates of ATT. The black lines represent the point estimates and gray shadows represnt 95\% confidence interval.}}
    \label{fig:att_prod_Myanmar}
\end{figure}

\begin{figure}[H] 
    \centering
    \caption{Impacts of Myanmar's Export Ban on Its Trade Partners' Roundwood Import Volume}
    \subfigure[][Coniferous]{\includegraphics*[bb=0 0 650 650, scale=0.2]{"~/OneDrive - The University of Tokyo/deforestation/fig/syntheticDID/Myanmar/ATT/imp_quant_Coniferous.png"}}
    \subfigure[][Non-conif Non-tropical]{\includegraphics*[bb=0 0 650 650, scale=0.2]{"~/OneDrive - The University of Tokyo/deforestation/fig/syntheticDID/Myanmar/ATT/imp_quant_Non-coniferous_non-tropical.png"}}
    \subfigure[][Non-conif Tropical]{\includegraphics*[bb=0 0 650 650, scale=0.2]{"~/OneDrive - The University of Tokyo/deforestation/fig/syntheticDID/Myanmar/ATT/imp_quant_Non-coniferous_tropical.png"}}
    \caption*{\small{The figure shows impacts of Myanmar's export restriction policy in 2014 on their trade partners' import volumne of coniferous roundwood (a), non-coniferous non-tropical roundwood (b), and non-coniferous tropical roundwood (c). X axis represents ellapesed time from the policy intervention. Thus, t = 0 means 2014. Y axis is the estimates of ATT. The black lines represent the point estimates and gray shadows represnt 95\% confidence interval.}}
    \label{fig:att_imp_Myanmar}
\end{figure}


\begin{figure}[H] 
    \centering
    \caption{Impacts of Myanmar's Export Restriction on Its Trade Partners' Roundwood Production}
    \subfigure[][Coniferous]{\includegraphics*[bb=0 0 900 700, scale=0.2]{"~/OneDrive - The University of Tokyo/deforestation/fig/syntheticDID/Myanmar/CF/Production_Industrial_roundwood__coniferous.png"}}
    \subfigure[][Non-coniferous]{\includegraphics*[bb=0 0 900 700, scale=0.2]{"~/OneDrive - The University of Tokyo/deforestation/fig/syntheticDID/Myanmar/CF/Production_Industrial_roundwood__non-coniferous.png"}}
    \caption*{\small{The figure shows impacts of Russia's export restriction policy in 2007 on their trade partners' production volumne of coniferous roundwood (a) and non-coniferous roundwood. X axis represents ellapesed time from the policy intervention. Thus, t = 0 means 2007. Y axis is the estimates of ATT. The black lines represent the point estimates and gray shadows represnt 95\% confidence interval.}}
    \label{fig:cf_prod_Myanmar}
\end{figure}

\begin{figure}[H] 
    \centering
    \caption{Impacts of Myanmar's Export Restriction on Its Trade Partners' Roundwood Import Volume}
    \subfigure[][Coniferous]{\includegraphics*[bb=0 0 650 650, scale=0.2]{"~/OneDrive - The University of Tokyo/deforestation/fig/syntheticDID/Myanmar/CF/imp_quant_Coniferous.png"}}
    \subfigure[][Non-conif Non-tropical]{\includegraphics*[bb=0 0 650 650, scale=0.2]{"~/OneDrive - The University of Tokyo/deforestation/fig/syntheticDID/Myanmar/CF/imp_quant_Non-coniferous_non-tropical.png"}}
    \subfigure[][Non-conif Tropical]{\includegraphics*[bb=0 0 650 650, scale=0.2]{"~/OneDrive - The University of Tokyo/deforestation/fig/syntheticDID/Myanmar/CF/imp_quant_Non-coniferous_tropical.png"}}
    \caption*{\small{The figure shows impacts of Russia's export restriction policy in 2007 on their trade partners' import volumne of coniferous roundwood (a), non-coniferous non-tropical roundwood (b), and non-coniferous tropical roundwood (c). X axis represents ellapesed time from the policy intervention. Thus, t = 0 means 2007. Y axis is the estimates of ATT. The black lines represent the point estimates and gray shadows represnt 95\% confidence interval.}}
    \label{fig:cf_imp_Myanmar}
\end{figure}

\section{Impacts on Trade Competitors}
This section discusses how Russia's log export restrictions and Myanmar's log export ban affected their trade competitors, respectively. For instance, Myanmar exported large number of teak logs (non-coniferous and tropical roundwood) to China, and Papua New Guinea was the other supplier of roundwood to China in 2013. Then, how did Papua New Guinea react to the Myanmar's supply shock in 2014? 

The quantity competition model predicts that Myanmar's trade competitors, such as Papua New Guinea, would also increase production of non-coniferous roundwood to supply to Myanmar's previous export destinations, such as China. At the same time, the model also implicates that the increase in production is not so large in Myanmar's trade competitors as in Myanmar's export destination countries due to trade costs. To empirically test the impacts of Russia's (Myanmar's) export restriction policy on their trade competitors, I apply shift-share approach which is common in literature of international trade and shock propagation in supply chain networks. In my specification, I follow the approach of \cite{huneeus2018production}. First, I describe the identification strategy, and then presents the estimation results.

\subsection{Identification Strategy}
The basic idea of shift-share design is to regress the outcome on how much the unit was exposed to the shock. In this setting, countries who supplied coniferous roundwood to markets where Russia supplied coniferous roundwood as well before the export restriction were drastically exposed to the supply shock of Russian conoferous roundwood. On the other hand, countries that did not export coniferous roundwood at all or did not export coniferous roundwood to the market where Russia supplied were not exposed to the supply shock in 2014 so much. To quantify this exposure, we define the exposure index as follows:
\begin{align}
    \text{Exposure}_i =  - \sum_{m \in N \setminus \{ i \}} \frac{v_{i \to m, t_0}}{\sum_{j \in N \setminus \{ i \}} v_{i \to j, t_0}} (v_{K \to m, t_0+2} - v_{K \to m, t_0+1})
\end{align}
where $v_{i \to m, t}$ denotes export value of roundwood from country $i$ to country $m$, $t_0$ denotes one year before the supply shock ($t_0 = 2006$ for Russia's case and $t_0 = 2013$ for Myanmar's case, respectively), and $K$ denotes the country which leads the supply shock (Russia or Myanmar). The first term, what is called ``share'', is a ratio of country $i$'s export value to country $m$ out of country $i$'s total export value before the shock began. Thus, if country $i$ was dependent on country $m$ as their supply destination to a great extent, then share of country $m$ for country $i$ is large. The second term, what is called ``shift'', is a difference in exports from shock leading country $K$ to country $m$ between one year after the shock and the start year of the shock (difference between 2008 and 2007 for Russian case and difference between 2015 and 2014 for Myanmar case, respectively). If country $m$ is a large customer for shock-leading country $K$ and the trade flow between these two country drops rapidly by the country $K$'s export restriction policy, then this shift takes a large negative value. Finally, to make it easy to interpret the estimation results, I reverse the sign of the exposure index. Thus, the positive coefficient of this exposure index means that the supply shock increases value of the outcome while negative coefficient means that the supply shock reduces value of the outcome. While suppliers endogeneously choose share value, which represents the destination of exports, the shift value, which represnts the shock-leading country's reduction in export due to its domestic reason, is exogeneous for the other suppliers, given the share. This combination of endogeneous share and exogeneous shift is consistent with what \cite{adao2019shift} validates. In addition, to avoid concerns that countries choose their supply destination by predicting the forthcoming Russia's (Myanmar's) supply shock, I use the lagged share of each markets. The qualitative evidence I present in section 3, the other countries could not predict the supply shock caused by the forthcoming export restriction policy one year before. Moreover, as I confirm in the descriptive statistics part, the links of the export is quite stable during periods without shocks. Thus, this lagged share still works as a share of the year when the shock started and it satisfies the assumption that the lagged share is not a result of prediction of forthcoming supply shocks.\\

In the regression, I regress the first difference in the outcome on the exposure indices of exports of coniferous roundwood, non-coniferous non-tropical roundwood, and non-coniferous tropical roundwood. I run the following regression formula:
\begin{align}
    y_{i, t_0 + h} - y_{i, t_0} = \beta_0 + \beta_1 &\text{Exposure}_{\text{Conif}, i} + \beta_2 \text{Exposure}_{\text{Non-conif Non-trop}, i} \nonumber \\
    &+ \beta_3 \text{Exposure}_{\text{Non-conif Trop}, i} + \varepsilon_{i, h}
\end{align}
where $h$ means length of the lag ($h = -5, -4, ..., 12$ for Russia's case and $h = -12, -11, ..., 5$ for Myanmar's case, respectively). $y_{i, t_0 + h} - y_{i, t_0}$ represents difference in the outcome between outcome of one year before the shock ($t_0$) and outcome of the current year ($t_0 + h$). I run this regression by year, namely by $h$.The outcomes I mainly presents hereafter are production of roundwoods and total exports of roundwoods of each country. Thus, the positive coefficient on the exposure index of the shocked roundwood (coniferous roundwood for Russia and non-coniferous non-tropical roundwood for Myanmar) indicates that the supply shock increases Russia's (Myanmar's) export competitors' production (export).

\subsection{Estimation Results}

\begin{figure}[H] 
    \centering
    \caption{Impacts of Russia's Export Restriction on Its Trade Partners' Roundwood Import Volume}
    \subfigure[][Coniferous]{\includegraphics*[bb=0 0 650 650, scale=0.15]{"~/OneDrive - The University of Tokyo/deforestation/fig/competitor/Russia/conif_exp_quant.png"}}
    \subfigure[][Non-conif Non-tropical]{\includegraphics*[bb=0 0 650 650, scale=0.15]{"~/OneDrive - The University of Tokyo/deforestation/fig/competitor/Russia/nonconif_nontrop_exp_quant.png"}}
    \subfigure[][Non-conif Tropical]{\includegraphics*[bb=0 0 650 650, scale=0.15]{"~/OneDrive - The University of Tokyo/deforestation/fig/competitor/Russia/nonconif_trop_exp_quant.png"}}
    \caption*{\small{The figure shows impacts of Russia's export restriction policy in 2007 on their trade partners' import volumne of coniferous roundwood (a), non-coniferous non-tropical roundwood (b), and non-coniferous tropical roundwood (c). X axis represents ellapesed time from the policy intervention. Thus, t = 0 means 2007. Y axis is the estimates of ATT. The black lines represent the point estimates and gray shadows represnt 95\% confidence interval.}}
    \label{fig:competitor_exp_Russia}
\end{figure}

\begin{figure}[H] 
    \centering
    \caption{Impacts of Russia's Export Restriction on Its Trade Partners' Roundwood Import Volume}
    \subfigure[][Coniferous]{\includegraphics*[bb=0 0 650 650, scale=0.15]{"~/OneDrive - The University of Tokyo/deforestation/fig/competitor/Russia/conif_Production.png"}}
    \subfigure[][Non-coniferous]{\includegraphics*[bb=0 0 650 650, scale=0.15]{"~/OneDrive - The University of Tokyo/deforestation/fig/competitor/Russia/nonconif_Production.png"}}
    \caption*{\small{The figure shows impacts of Russia's export restriction policy in 2007 on their trade partners' import volumne of coniferous roundwood (a), non-coniferous non-tropical roundwood (b), and non-coniferous tropical roundwood (c). X axis represents ellapesed time from the policy intervention. Thus, t = 0 means 2007. Y axis is the estimates of ATT. The black lines represent the point estimates and gray shadows represnt 95\% confidence interval.}}
    \label{fig:competitor_prod_Russia}
\end{figure}


\begin{figure}[H] 
    \centering
    \caption{Impacts of Russia's Export Restriction on Its Trade Partners' Roundwood Import Volume}
    \subfigure[][Coniferous]{\includegraphics*[bb=0 0 650 650, scale=0.15]{"~/OneDrive - The University of Tokyo/deforestation/fig/competitor/Myanmar/conif_exp_quant.png"}}
    \subfigure[][Non-conif Non-tropical]{\includegraphics*[bb=0 0 650 650, scale=0.15]{"~/OneDrive - The University of Tokyo/deforestation/fig/competitor/Myanmar/nonconif_nontrop_exp_quant.png"}}
    \subfigure[][Non-conif Tropical]{\includegraphics*[bb=0 0 650 650, scale=0.15]{"~/OneDrive - The University of Tokyo/deforestation/fig/competitor/Myanmar/nonconif_trop_exp_quant.png"}}
    \caption*{\small{The figure shows impacts of Russia's export restriction policy in 2007 on their trade partners' import volumne of coniferous roundwood (a), non-coniferous non-tropical roundwood (b), and non-coniferous tropical roundwood (c). X axis represents ellapesed time from the policy intervention. Thus, t = 0 means 2007. Y axis is the estimates of ATT. The black lines represent the point estimates and gray shadows represnt 95\% confidence interval.}}
    \label{fig:competitor_exp_Myanmar}
\end{figure}


\begin{figure}[htbp] 
    \centering
    \caption{Impacts of Russia's Export Restriction on Its Trade Partners' Roundwood Import Volume}
    \subfigure[][Coniferous]{\includegraphics*[bb=0 0 650 650, scale=0.15]{"~/OneDrive - The University of Tokyo/deforestation/fig/competitor/Myanmar/conif_Production.png"}}
    \subfigure[][Non-coniferous]{\includegraphics*[bb=0 0 650 650, scale=0.15]{"~/OneDrive - The University of Tokyo/deforestation/fig/competitor/Myanmar/nonconif_Production.png"}}
    \caption*{\small{The figure shows impacts of Russia's export restriction policy in 2007 on their trade partners' import volumne of coniferous roundwood (a), non-coniferous non-tropical roundwood (b), and non-coniferous tropical roundwood (c). X axis represents ellapesed time from the policy intervention. Thus, t = 0 means 2007. Y axis is the estimates of ATT. The black lines represent the point estimates and gray shadows represnt 95\% confidence interval.}}
    \label{fig:competitor_prod_Myanmar}
\end{figure}


\section{Discussion}

\section{Conclusion}

\bibliography{deforestation_reference}

\section*{Appendix}


\begin{table}[htbp]
    \caption{Cross-sectional Summary Statistics on Roundwood Price in 2018}
    \centering
    \begin{tabular}{lcccccc}
        \tabularnewline \midrule \midrule
        variable & n & mean & sd & min & median & max\\
        \midrule
        \emph{Import Price (1000USD / $m^3$)}\\
        Coniferous & 86 & 0.059 & 0.009 & 0.000 & 0.061 & 0.062\\
        Non-coniferous Non-tropical & 104 & 0.327 & 0.529 & 0.000 & 0.179 & 4.400\\
        Non-coniferous Tropical & 59 & 0.879 & 1.505 & 0.000 & 0.571 & 10.750\\
        \midrule
        \emph{Export Price (1000USD / $m^3$)}\\
        Coniferous & 86 & 0.051 & 0.010 & 0.000 & 0.053 & 0.060\\
        Non-coniferous Non-tropical & 104 & 0.457 & 1.071 & 0.000 & 0.204 & 9.562\\
        Non-coniferous Tropical & 59 & 3.349 & 18.030 & 0.031 & 0.520 & 138.733\\
\midrule \midrule
        \end{tabular}
        \caption*{\small{Summary statistics of import price and export price in roundwood markets. The samples are countries with positive quantity of import or export of roundwoods.}}
        \label{tab:summary_stat_price}
 \end{table}


\subsection*{A. Gravity Equation}
\begin{figure}[htbp] 
    \centering
    \caption{Impacts of Russia's Export Restriction on Its Trade Partners' Roundwood Import Volume}
    \subfigure[][Coniferous]{\includegraphics*[bb=0 0 900 650, scale=0.15]{"~/OneDrive - The University of Tokyo/deforestation/fig/gravity_eq/Industrial_roundwood__coniferous__export_import_.png"}}
    \subfigure[][Non-coniferous]{\includegraphics*[bb=0 0 900 650, scale=0.15]{"~/OneDrive - The University of Tokyo/deforestation/fig/gravity_eq/Industrial_roundwood__non-coniferous_non-tropical__export_import_.png"}}
    \subfigure[][Non-coniferous]{\includegraphics*[bb=0 0 900 650, scale=0.15]{"~/OneDrive - The University of Tokyo/deforestation/fig/gravity_eq/Industrial_roundwood__non-coniferous_tropical__export_import_.png"}}
    \caption*{\small{The figure shows impacts of Russia's export restriction policy in 2007 on their trade partners' import volumne of coniferous roundwood (a), non-coniferous non-tropical roundwood (b), and non-coniferous tropical roundwood (c). X axis represents ellapesed time from the policy intervention. Thus, t = 0 means 2007. Y axis is the estimates of ATT. The black lines represent the point estimates and gray shadows represnt 95\% confidence interval.}}
    \label{fig:competitor_prod_Myanmar}
\end{figure}


\end{document}