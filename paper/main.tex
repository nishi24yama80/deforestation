\documentclass[a4paper,12pt]{article}


\usepackage{amssymb,amsmath,amsfonts,eurosym,geometry,ulem,caption,color,setspace,sectsty,comment,caption,pdflscape,subfigure,array,hyperref, here}


\usepackage{bm}
% figure
\usepackage[dvipdfmx]{graphicx}
%\usepackage{bmpsize}

\usepackage{booktabs}
\usepackage{siunitx}
\newcolumntype{d}{S[
    input-open-uncertainty=,
    input-close-uncertainty=,
    parse-numbers = false,
    table-align-text-pre=false,
    table-align-text-post=false
 ]}


\usepackage[bottom]{footmisc}

\usepackage[round]{natbib}
\bibliographystyle{plainnat}

\begin{document}

\title{Oasis or Mirage? \\ Trade Network Propagation of Deforestation}
\author{Tomoki Nishiyama\footnote{Department of Economics, the University of Tokyo. \\ e-mail address: nishiyama-tomoki@g.ecc.u-tokyo.ac.jp \\ I thank Yasuyuki Sawada (an advisor of this thesis), Michihiro Kandori, and all the participants of their seminar for helpful discussion. I am also grateful to Shotaro Beppu, Yutao Chen, and Yuto Iwamoto for meaningful comments and supports.}}
\date{January 2024}
\maketitle

\begin{abstract}
    Log production is destroying forests around the world and contributing to global warming. To address deforestation, various countries have begun to adopt policies to restrict log exports. However, even if one country restricts log exports, the countries that have been importing logs may increase their log production to compensate for their previous imports. It is also possible that other countries will increase their log exports to the log consuming countries. In this case, one country's log export restrictions could induce deforestation in other countries through trade networks. In this paper, we focus on two policies, Russia's log export restrictions in 2007 and Myanmar's log export ban in 2014, to demonstrate how these policies affect the log industry in other countries through trade networks. We find that when one country restricts log exports, log production increases in their previously main export destination. This result suggests that through a network of trade, one country's pro-environmental policies can cause environmental damage in other countries.
\end{abstract}

\textit{
    A king is wandering in the desert. There used to be a forest in this desert. However, due to the mass production of logs, the former forest has been lost and has become a desert. The once cool forest is now a scorching desert with no water to be found no matter how far one walks. The king, who is about to collapse, finds something like water in the distance. The king says, "There is an oasis spreading out over there. There must be water. There must be trees. We can regenerate the forest from there." And he starts running. But, is that water really an oasis? Is it not a mirage?
}
\section{Introduction}
Deforestation is an urgent issue on this planet: 50\% of forests has already been lost since 1990 (DATA). Deforestation is causing devastation of the global environment. Deforestation accounts for 12-20\% of global greenhouse gas emissions, as it reduces the capacity to absorb carbon dioxide through plants' photosynthesis and contributes to climate change. Deforestation also causes loss of biodiversity because forests are home to a wide variety of plants and animals. The extraction of timber to produce forest products is one of the main causes of deforestation. To address this serious problem, some countries have banned log exports since the 1980s. For example, Myanmar, which has destroyed half of its tropical forests since 1960 and worked as the world's third largest supplier (DATA), banned log exports in 2014. \\

However, to measure the impact of environmental policies that target global issues, it is necessary to discuss how those policies affect other countries. Even a pro-environmental policy that protect forests in one's own country may cause new deforestation in other countries, thereby weakening the effectiveness of the policy on a global scale. To gain insight into this issue, this paper investigates how a country's forest protection policy can propagate deforestation to its trading partners through trade networks. I answer this critical question by examining two policies, Russia's restriction on log exports in 2007 and Myanmar's prohibition of log exports in 2014, respectively to check the impacts on the other countries' log production and exports.\\

Let us consider a simple case study. Myanmar prohibited their log exports in 2014. Until 2014, Myanmar was the third largest suppliers of timbers in the world. How would pro-environmental policy of major log supplies affect the other countries? First, trade partner countries, such as China or India, who imported lots of logs from Myanmar might extract more timbers in their own forests to compensate the supply shocks in the domestic log market. As a result, forests in Myanmar's export destination might be reduced due to Myanmar's pro-environmental policy. Second, the other countries, such as Congo or Indonesia, who shared the same export destination as Myanmar, might increase their exports to such destinations to compensate the reduction in Myanmar's exports. Then, forests in these supplier countries might be exploited due to Myanmar's export ban policy. As these stories, one country's policy to protect their own forest induces the other countries' deforestation through trade flow networks. Then, such network spillovers weaken the worldwide effect of the pro-environmental policy.\\

To empirically discuss this network propagation, I examine two historical policies, the 2007 Russian log export restriction and the 2014 Myanmar log export ban, respectively, and measure their effects on log production and exports in other countries. These two policy interventions have several useful features that make them ideal natural experiments for studying spillover effects. First, Russia and Myanmar were major suppliers of logs until their exports were restricted. Russia developed the largest coniferous forest in the world, the Siberian "taiga", which accounted for half of the world's supply of softwood lumber. Myanmar logged half of its tropical rainforests and exported teak logs to the world, accounting for 10\% of the world's tropical log supply. Therefore, a policy of export restrictions in these two major supplying countries would cause major supply shocks to other countries. Second, for countries other than these two, these policies were unpredictable exogenous shocks. Thus, the characteristics of these policies allow us to examine how environmental protection policies in one country affect forestry in other countries.\\

The remainder of this paper is organized as follows. Section 2 provides an overview of two trade restriction policies (Russia in 2007 and Myanmar in 2014) and explains how these policies are exogeneous shocks for other countries. Section 3 describes how I construct trade network flow data for forest products, and Section 4 presents a sketch of the network data to give an overall picture of the trade network. Based on some basic facts explained in Section 4, Section 5 presents a simple model for constructing hypotheses on how one country's policy affects other countries. Section 6 describes the identification strategy for empirically testing the theoretical hypothesis. Section 7 presents the empirical results and Section 8 discusses their interpretation. Finally, Section 9 presents the conclusions. Extensions of the theoretical model and robustness checks of the empirical results are in the appendix.\\

\cite{acemoglu2012network}

\section{Russia and Myanma's Restriction Policy}

\section{Trade Flow Data}

\section{Sketch of the Trade Network of Logs}

\section{A Model of Deforestation Propagation}

\section{identification Strategy}

\section{Estimation Results of Network Propagation}

\section{Discussion}

\section{Conoclusion}

\bibliography{deforestation_reference}

\end{document}